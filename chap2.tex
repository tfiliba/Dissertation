\chapter{Real Time Radio Astronomy Instrumentation}
%Summary: Give an description of radio astronomy instrumentation \\
%Describe science goals and mathematical function \\
%Goal: Give background for the application. Show that most instruments use a small number of high level functions\\
%State: Can be written now \\
%Use info from casper papers and radio 101 \\
%References already compiled \\

Radio telescopes produce very high amounts of data. 
The reason for this high influx of data is twofold. 
First, to enable new science, new radio antenna observe increasingly higher bandwidths of data. 
Second, to sate the need for larger collecting area, rather than designing a large single dish, many new telescopes are being designed as antenna arrays, where the data from multiple antennas is combined to act as a single large dish.
As the size of the arrays and bandwidths for single dishes simultaneously increase, the data produced cannot be feasibly recorded in real-time. 
To cope with the progress science and antenna technology, there is a constant need for new systems to process, rather than record, this data in real time.
Once the data is partially processed, and reduced to a manageable bandwidth, it can be stored and processed further offline, where there is no longer a need for low-latency, high-bandwidth hardware.


\section{Real Time Algorithms}
There is a small number of real time algorithms commonly used to reduce the data, in this section we focus on spectroscopy, pulsar processing, beamforming and interferometry. 


\subsection{Spectroscopy}
A spectrometer is simply an instrument that produces an integrated spectrum from a time domain signal. After digitizing the data, a spectrum is constantly computed (channelization), then each channel is summed for a predetermined amount of time to compute the average power in that channel. Figure (TODO: add spectrometer block diagram) shows a block diagram for a simple spectrometer design. After digitization, there is the channelization step, where the signal is processed by a digital filter bank, and then the channels are accumulated. 

High-resolution, high-bandwidth spectrometers often require more complexity in their design. Once the number of channels is sufficiently high, it becomes infeasible to compute the spectrum using a single filter bank. To cope with this, the channelization is done in two steps as shown in figure (TODO: add hi res spectrometer block diagram). In the first step, the signal is divided into coarse channels using a filter bank. At this point the channels are much wider than intended and can't be accumulated yet. After coarse channelization, the spectrometer treats the data from a single channel as time domain data and passes it through a filter bank again. This step breaks up the wide channel into a number of smaller channels . At this point the data can be accumulated since it has the desired resolution.

\subsection{Pulsar processing}
A pulsar processor is an instrument designed specifically to observe transient events, such as pulsars. A pulsar is a 
%Detect dispersed pulses
%
%\subsection{Beamforming}
%%Add together multiple (delayed) signals to improve SNR
%\subsection{Interferometry}
%%Form an image
%An interferometer 

\section{Science Goals}
Radio astronomy simply refers to the type of science that can be done by observing astronomical objects at radio wavelengths, rather than a specific scientific goal. 
There is a huge variety of different experiments, such as searching for gravity waves (TODO: ref nangrav), traces of the first stars (TODO: ref PAPER), or aliens (TODO: ref SETI).
But, despite this variety, the small number of algorithms detailed above serve as the first step in processing the data for many such projects.

SETI, or the Search for Extra Terrestrial Intelligence, 

