\chapter{Related Work}
%Summary: 2 major sections, 1 about past radio instrumentation, 1 about automatic mapping\\
%Goal: Explain why we can mesh them together successfully in this case\\
%State: Can be written now \\
%List of references already compiled into Papers archive \\
\section{Radio Astronomy} \label{Related Work:Radio Astronomy}
\subsection{Digital Signal Processing for Radio Astronomy}
\subsection{DiFX}
%DiFX (Distributed FX Correlator) is a scalable software implementation
%Designed for VLBI (Very Long Baseline Interferometry)
%Requires a large cluster to do a lot of computation
%64 MHz, 10 antennas required ~100 nodes in 2007
%A. T Deller, S. J Tingay, M Bailes, and C West. Difx: A software correlator for very long baseline interferometry using multiprocessor computing environments. Publications of the Astronomical Society of the Pacific, 119(853):318�336, Mar 2007.
\subsection{LOFAR}
%R Van Nieuwpoort. Correlating Radio Astronomy Signals with Many-Core Hardware. International Journal of Parallel Programming, 2009.
\subsection{xGPU}
%M Clark and P La Plante. Accelerating Radio Astronomy Cross-Correlation with Graphics Processing Units. Arxiv preprint arXiv:1107.4264, 2011.
\subsection{CASPER}
%A Parsons, et al. PetaOp/Second FPGA Signal Processing for SETI and Radio Astronomy. Signals, Systems and Computers, 2006. ACSSC �06. Fortieth Asilomar Conference on, pages 2031�2035, 2006.

%CASPER = Collaboration for Astronomy Signal Processing and Electronics Research
%Identifies commonly used DSP blocks for radio astronomy
%FFTs (tunable bandwidth, number of channels, real or complex)
%Polyphase filter banks
%Accumulators
%Digital downconverters/mixers
%Scripts are used to configure parameterized blocks

%Modular Hardware
%Low number of FPGA board designs
%Can be upgraded piecemeal or all together
%Standard signal processing model which is consistent between upgrades

%Boards work with a variety of ADCs/DACs etc.
%Dual input 8 bit 1 Gsps (or 2 Gsps single input interleaved)
%3 Gsps single input 8 bit ADC (interleave 2 for 6 Gsps)
%Provides tested hardware for new instruments

%Aaron Parsons, et al.  A Scalable Correlator Architecture Based on Modular FPGA Hardware, Reuseable Gateware, and Data Packetization. Publications of the Astronomy Society of the Pacific, 120:1207�1221, September 2008.



\section{Tuning} \label{Related Work:Tuning}
\subsection{Metropolis}
%Abhijit Davare. Automated mapping for heterogeneous multiprocessor embedded systems. PhD thesis, 2007.
 
%Provides an abstract block based description of the system (Metropolis)
%Easy to stitch algorithms together 
%Simulation is abstracted from the eventual hardware implementation
%Mapping is focused on scheduling onto heterogeneous platforms
%Strong focus on embedded systems

%Some people have thought about how to use technology
%Metropolis works on simulation extensively (supporting 
%Embedded systems � we have 1 cpu and 1 dsp let�s make it go fast
%This doesn�t solve our problem
%Just does scheduling

%Doesn�t help design the cluster
%A �heterogeneous node� has a fixed mix of resources
%Can�t reduce costs by throwing away certain types of hardware
%Optimization is based on a fixed architecture and flexible performance
%Doesn�t match our �always running� model
%Performance is �good enough�, not an optimization target



%Lisa Marie Guerra (?)


\subsection{ILP for scheduling}
An integer linear programming model for mapping applications on hybrid systems
