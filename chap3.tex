\chapter{Related Work}
%Summary: 2 major sections, 1 about past radio instrumentation, 1 about automatic mapping\\
%Goal: Explain why we can mesh them together successfully in this case\\
%State: Can be written now \\
%List of references already compiled into Papers archive \\
\section{Radio Astronomy}
\subsection{Digital Signal Processing for Radio Astronomy}
\subsection{DiFX}
%DiFX (Distributed FX Correlator) is a scalable software implementation
%Designed for VLBI (Very Long Baseline Interferometry)
%Requires a large cluster to do a lot of computation
%64 MHz, 10 antennas required ~100 nodes in 2007
\subsection{LOFAR}
\subsection{xGPU}

\subsection{CASPER}

\section{Tuning}
\subsection{Metropolis}
Abhijit Davare. Automated mapping for heterogeneous multiprocessor embedded systems. PhD thesis, 2007.
 
Provides an abstract block based description of the system (Metropolis)
Easy to stitch algorithms together 
Simulation is abstracted from the eventual hardware implementation
Mapping is focused on scheduling onto heterogeneous platforms
Strong focus on embedded systems

Some people have thought about how to use technology
Metropolis works on simulation extensively (supporting 
Embedded systems � we have 1 cpu and 1 dsp let�s make it go fast
This doesn�t solve our problem
Just does scheduling

Doesn�t help design the cluster
A �heterogeneous node� has a fixed mix of resources
Can�t reduce costs by throwing away certain types of hardware
Optimization is based on a fixed architecture and flexible performance
Doesn�t match our �always running� model
Performance is �good enough�, not an optimization target



Lisa Marie Guerra (?)

\subsection{ILP for scheduling}
An integer linear programming model for mapping applications on hybrid systems
